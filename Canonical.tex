\documentclass[12pt]{article}
\usepackage[utf8]{inputenc}
\usepackage[headings]{fullpage}
\usepackage{ocamlweb}
\pagestyle{headings}

\usepackage[T1]{fontenc}
\usepackage{amsmath}
\usepackage{amsfonts}
\usepackage{amssymb}

\begin{document}
%%%%%%%%%%%%%%%%%%%%%%%%%%%%%%%%%%%%%%%%%%%%%%%%%%%%%%%%%%%%%%%%%
%% This file has been automatically generated with the command
%% ocamlweb --header --no-index Canonical.ml -o Canonical.tex 
%%%%%%%%%%%%%%%%%%%%%%%%%%%%%%%%%%%%%%%%%%%%%%%%%%%%%%%%%%%%%%%%%
\typeout{OcamlWeb file Canonical.ml}
\ocwmodule{Canonical}
\allowbreak\ocwsection
\label{Canonical.ml:0}%
\label{Canonical.ml:97}%
\medskip
\ocwbegincode{}\ocwindent{0.00em}
\ocwkw{module}~$\ocwupperid{P}~=~\ocwupperid{Permutation}$\medskip

\ocwendcode{}\ocwindent{0.00em}
Type décrivant une tresse représentée sous la forme $\Delta^r A_1...A_n$ où
   $A_1,...,A_n$ sont des tresses simples. Les tresses simples sont ici décrites
   par leurs permutations associées.

\ocweol
\label{Canonical.ml:330}%
\medskip
\ocwbegincode{}\ocwindent{0.00em}
\ocwkw{type}~$\ocwlowerid{braid\_permlist}~=~\{\ocwlowerid{bpl\_size}~:~$\ocwbt{int};~$\ocwlowerid{delta\_power}~:~$\ocwbt{int};~$\ocwlowerid{permlist}~:~\ocwupperid{P.}\ocwlowerid{permutation}~$\ocwbt{list}\}\medskip

\ocwendcode{}\ocwindent{0.00em}
Tresses de base : $\Delta$ et la tresse vide

\ocweol
\label{Canonical.ml:473}%
\medskip
\ocwbegincode{}\ocwindent{0.00em}
\ocwkw{let}~$\ocwlowerid{delta}~\ocwlowerid{n}~=~\{\ocwlowerid{bpl\_size}~=~\ocwlowerid{n};~\ocwlowerid{delta\_power}~=~1;~\ocwlowerid{permlist}~=~[\,]\}$\medskip

\label{Canonical.ml:536}%
\ocwindent{0.00em}
\ocwkw{let}~$\ocwlowerid{empty}~\ocwlowerid{n}~=~\{\ocwlowerid{bpl\_size}~=~\ocwlowerid{n};~\ocwlowerid{delta\_power}~=~0;~\ocwlowerid{permlist}~=~[\,]\}$\ocweol
\ocwendcode{}\allowbreak\ocwsection
\label{Canonical.ml:602}%
Conversion d'une tresse décrit par un mot sur l'alphabet des générateurs
   en une représentation sous forme de liste de permutations.
   
   Algorithme : Si l'on rencontre un $\sigma _i$ alors on ajoute à la liste des
   permutations la transposition correspondante.
   Si l'on rencontre un ${\sigma _i}^{-1}$, on utilise le fait que
   ${\sigma _i}^{-1} = \Delta^{-1} \Delta {\sigma _i}^{-1}$ et que $\Delta {\sigma _i}^{-1}$
   est une tresse positive. On l'ajoute à la liste et on fait remonter le $\Delta^{-1}$
   en composant par $\tau$ .
   
   On simplifie les calculs en se rappelant que $\tau^2 = id$.

\ocweol
\label{Canonical.ml:1226}%
\medskip
\ocwbegincode{}\ocwindent{0.00em}
\ocwkw{let}~$\ocwlowerid{get\_permlist\_decomposition}~(\ocwlowerid{b}~:~\ocwupperid{Braid.}\ocwlowerid{braid})~=~$\ocweol
\ocwindent{2.00em}
\ocwkw{let}~$\ocwlowerid{n}~=~\ocwlowerid{b.Braid.}\ocwlowerid{size}~$\ocwkw{in}\ocweol
\ocwindent{2.00em}
\ocwbc{} Calcule $\Delta \times \sigma^{-1}$ que l'on sait être un facteur canonique \ocwec{}\ocweol
\ocwindent{2.00em}
\ocwkw{let}~$\ocwlowerid{delta}~=~\ocwupperid{P.}\ocwlowerid{make\_delta}~\ocwlowerid{n}~$\ocwkw{in}\ocweol
\ocwindent{2.00em}
\ocwkw{let}~$\ocwlowerid{neg\_to\_perm}~\ocwlowerid{x}~=~\ocwupperid{P.}\ocwlowerid{compose\_transpose\_left}~\ocwlowerid{delta}~(-\ocwlowerid{x}-1)~(-\ocwlowerid{x})~$\ocwkw{in}\ocweol
\ocwindent{2.00em}
\ocwbc{} Applique l'algorithme, en mémorisant la puissance de $\tau$ à appliquer \ocwec{}\ocweol
\ocwindent{2.00em}
\ocwkw{let}~$(\ocwlowerid{delta\_pow},~\ocwlowerid{perm\_stack})~=~$\ocweol
\ocwindent{4.00em}
$\ocwupperid{List.}\ocwlowerid{fold\_left}~($\ocwkw{fun}~$(\ocwlowerid{delta\_pow},~\ocwlowerid{perm\_stack})~\ocwlowerid{x}~\rightarrow{}$\ocweol
\ocwindent{13.00em}
\ocwkw{if}~$\ocwlowerid{x}~>~0~$\ocwkw{then}\ocweol
\ocwindent{14.00em}
$(\ocwlowerid{delta\_pow},~$\ocweol
\ocwindent{14.00em}
$(\ocwupperid{P.}\ocwlowerid{make\_transpose}~(\ocwlowerid{x}-1)~\ocwlowerid{x}~\ocwlowerid{n},~0)::\ocwlowerid{perm\_stack})$\ocweol
\ocwindent{13.00em}
\ocwkw{else}\ocweol
\ocwindent{14.00em}
$(\ocwlowerid{delta\_pow}~-~1,$\ocweol
\ocwindent{14.50em}
$(\ocwlowerid{neg\_to\_perm}~\ocwlowerid{x},~0)::$\ocweol
\ocwindent{14.50em}
$(\ocwupperid{List.}\ocwlowerid{map}~($\ocwkw{fun}~$(\ocwlowerid{p},~\ocwlowerid{tau\_pow})~\rightarrow{}~(\ocwlowerid{p},~\ocwlowerid{tau\_pow}~+~1))~\ocwlowerid{perm\_stack})))$\ocweol
\ocwindent{11.50em}
$(0,~[\,])~\ocwlowerid{b.Braid.}\ocwlowerid{word}~$\ocwkw{in}\ocweol
\ocwindent{2.00em}
\ocwbc{} On applique la puissance de $\tau$ si nécessaire \ocwec{}\ocweol
\ocwindent{2.00em}
\ocwkw{let}~$\ocwlowerid{perm\_list}~=$\ocweol
\ocwindent{4.00em}
$\ocwupperid{List.}\ocwlowerid{rev\_map}~($\ocwkw{fun}~$(\ocwlowerid{perm},~\ocwlowerid{tau\_pow})~\rightarrow{}~$\ocwkw{if}~$(\ocwlowerid{tau\_pow}~$\ocwkw{mod}~2$)~\not=~0$\ocweol
\ocwindent{24.00em}
\ocwkw{then}~$\ocwupperid{P.}\ocwlowerid{tau}~\ocwlowerid{perm}$\ocweol
\ocwindent{24.00em}
\ocwkw{else}~$\ocwlowerid{perm})$\ocweol
\ocwindent{10.50em}
$\ocwlowerid{perm\_stack}~$\ocwkw{in}\ocweol
\ocwindent{2.00em}
\{$\ocwlowerid{bpl\_size}~=~\ocwlowerid{n};~\ocwlowerid{delta\_power}~=~\ocwlowerid{delta\_pow};~\ocwlowerid{permlist}~=~\ocwlowerid{perm\_list}\}$\ocweol
\ocwendcode{}\allowbreak\ocwsection
\label{Canonical.ml:2470}%
Mise sous forme maximale à gauche (left-weighted) d'une liste de permutations (et non
   d'une tresse complète : voir canonicize pour celà).
   
   Algorithme : On considère tour à tour des couples de tresses simples. On calcule alors
   le «finishing set» de la première et le «starting set» de la seconde. On fait la 
   différence ensembliste entre le «starting set» et le «finishing set» et l'on ajoute
   les éléments de la différence à la première tresse, et on les retire de la deuxième
   (concrètement, une composition de permutations).

\ocweol
\label{Canonical.ml:3044}%
\medskip
\ocwbegincode{}\ocwindent{0.00em}
\ocwkw{let}~$\ocwlowerid{make\_left\_weighted}~\ocwlowerid{start\_pl}~=$\medskip

\label{Canonical.ml:3079}%
\ocwindent{1.00em}
\ocwkw{let}~$\ocwlowerid{continue}~=~$\ocwbt{ref}~\ocwkw{true}~\ocwkw{in}\medskip

\ocwendcode{}\ocwindent{1.00em}
trouver la décomposition maximale à gauche de $A_1 A_2$,
     de permutations associées respectives p1 et p2 
\ocweol
\label{Canonical.ml:3231}%
\medskip
\ocwbegincode{}\ocwindent{1.00em}
\ocwkw{let}~\ocwkw{rec}~$\ocwlowerid{make\_lw\_pair}~\ocwlowerid{p1}~\ocwlowerid{p2}~=$\ocweol
\ocwindent{2.00em}
\ocwkw{let}~$\ocwlowerid{s2}~=~\ocwupperid{P.}\ocwlowerid{starting\_set}~\ocwlowerid{p2}~$\ocwkw{and}~$\ocwlowerid{f1}~=~\ocwupperid{P.}\ocwlowerid{finishing\_set}~\ocwlowerid{p1}~$\ocwkw{in}\ocweol
\ocwindent{2.00em}
\ocwkw{if}~$\ocwlowerid{s2}~=~[\,]~$\ocwbc{} p2 = id \ocwec{}~\ocwkw{then}~$[\ocwlowerid{p1}]$\ocweol
\ocwindent{2.00em}
\ocwkw{else}~\ocwkw{if}~$\ocwlowerid{f1}~=~[\,]~$\ocwbc{} p1 = id \ocwec{}~\ocwkw{then}~$[\ocwlowerid{p2}]~$\ocwbc{} peu probable \ocwec{}\ocweol
\ocwindent{2.00em}
\ocwkw{else}~\ocwkw{match}~$\ocwupperid{P.}\ocwlowerid{set\_difference}~\ocwlowerid{s2}~\ocwlowerid{f1}~$\ocwkw{with}\ocweol
\ocwindent{3.00em}
$\mid{}~[\,]~\rightarrow{}~[\ocwlowerid{p1};~\ocwlowerid{p2}]~$\ocwbc{} p1 facteur maximal \ocwec{}\ocweol
\ocwindent{3.00em}
$\mid{}~\ocwlowerid{i}::\ocwlowerid{\_}~\rightarrow{}~\ocwlowerid{continue}~:=~$\ocwkw{true};\ocweol
\ocwindent{8.00em}
\ocwkw{let}~$\ocwlowerid{p1'}~=~\ocwupperid{P.}\ocwlowerid{compose\_transpose\_left}~\ocwlowerid{p1}~(\ocwlowerid{i}-1)~\ocwlowerid{i}~$\ocwkw{in}\ocweol
\ocwindent{8.00em}
\ocwkw{let}~$\ocwlowerid{p2'}~=~\ocwupperid{P.}\ocwlowerid{compose\_transpose\_right}~\ocwlowerid{p2}~(\ocwlowerid{i}-1)~\ocwlowerid{i}~$\ocwkw{in}\ocweol
\ocwindent{8.00em}
$\ocwlowerid{make\_lw\_pair}~\ocwlowerid{p1'}~\ocwlowerid{p2'}$\ocweol
\ocwindent{1.00em}
\ocwkw{in}\medskip

\label{Canonical.ml:3700}%
\ocwindent{1.00em}
\ocwkw{let}~$\ocwlowerid{make\_lw\_pair'}~\ocwlowerid{p1}~\ocwlowerid{p2}~=$\ocweol
\ocwindent{2.00em}
\ocwkw{let}~$\ocwlowerid{b}~=~\ocwupperid{P.}\ocwlowerid{meet}~(\ocwupperid{P.}\ocwlowerid{compose}~(\ocwupperid{P.}\ocwlowerid{inv}~\ocwlowerid{p1})~(\ocwupperid{P.}\ocwlowerid{make\_delta}~(\ocwupperid{Array.}\ocwlowerid{length}~\ocwlowerid{p1})))~\ocwlowerid{p2}~$\ocwkw{in}\ocweol
\ocwindent{2.00em}
\ocwkw{if}~$\lnot{}~(\ocwupperid{P.}\ocwlowerid{is\_id}~\ocwlowerid{b})~$\ocwkw{then}~$($\ocweol
\ocwindent{3.00em}
$\ocwlowerid{continue}~:=~$\ocwkw{true};\ocweol
\ocwindent{3.00em}
$\ocwupperid{List.}\ocwlowerid{filter}~($\ocwkw{fun}~$\ocwlowerid{x}~\rightarrow{}~\lnot{}~(\ocwupperid{P.}\ocwlowerid{is\_id}~\ocwlowerid{x}))~[\ocwupperid{P.}\ocwlowerid{compose}~\ocwlowerid{p1}~\ocwlowerid{b};~\ocwupperid{P.}\ocwlowerid{compose}~(\ocwupperid{P.}\ocwlowerid{inv}~\ocwlowerid{b})~\ocwlowerid{p2}]$\ocweol
\ocwindent{2.00em}
$)~$\ocwkw{else}~$($\ocweol
\ocwindent{3.00em}
$\ocwupperid{List.}\ocwlowerid{filter}~($\ocwkw{fun}~$\ocwlowerid{x}~\rightarrow{}~\lnot{}~(\ocwupperid{P.}\ocwlowerid{is\_id}~\ocwlowerid{x}))~[\ocwlowerid{p1};~\ocwlowerid{p2}]$\ocweol
\ocwindent{2.00em}
$)$\ocweol
\ocwindent{1.00em}
\ocwkw{in}\medskip

\ocwendcode{}\ocwindent{1.00em}
On réduit la liste des facteurs à partir de la fin 
\ocweol
\label{Canonical.ml:4089}%
\medskip
\ocwbegincode{}\ocwindent{1.00em}
\ocwkw{let}~\ocwkw{rec}~$\ocwlowerid{reduce}~=~$\ocwkw{function}\ocweol
\ocwindent{2.00em}
$\mid{}~[\,]~\rightarrow{}~[\,]$\ocweol
\ocwindent{2.00em}
$\mid{}~\ocwlowerid{p1}::\ocwlowerid{q}~\rightarrow{}~$\ocwkw{match}~$\ocwlowerid{reduce}~\ocwlowerid{q}~$\ocwkw{with}\ocweol
\ocwindent{4.00em}
$\mid{}~[\,]~\rightarrow{}~[\ocwlowerid{p1}]$\ocweol
\ocwindent{4.00em}
$\mid{}~\ocwlowerid{p2}::\ocwlowerid{q'}~\rightarrow{}~\ocwlowerid{make\_lw\_pair'}~\ocwlowerid{p1}~\ocwlowerid{p2}~@~\ocwlowerid{q'}$\ocweol
\ocwindent{1.00em}
\ocwkw{in}\medskip

\label{Canonical.ml:4227}%
\ocwindent{1.00em}
\ocwkw{let}~$\ocwlowerid{current\_pl}~=~$\ocwbt{ref}~$\ocwlowerid{start\_pl}~$\ocwkw{in}\ocweol
\ocwindent{1.00em}
\ocwkw{while}~!$\ocwlowerid{continue}~$\ocwkw{do}\ocweol
\ocwindent{2.00em}
$\ocwlowerid{continue}~:=~$\ocwkw{false};\ocweol
\ocwindent{2.00em}
$\ocwlowerid{current\_pl}~:=~\ocwlowerid{reduce}~!\ocwlowerid{current\_pl}$\ocweol
\ocwindent{1.00em}
\ocwkw{done};\ocweol
\ocwindent{1.00em}
!$\ocwlowerid{current\_pl}$\ocweol
\ocwendcode{}\allowbreak\ocwsection
\label{Canonical.ml:4371}%
Met une tresse décrite sous forme de liste de permutations sous forme canonique.
   
   Algorithme : Rend la liste de tresses simples maximale à gauche, et collecte
   les tresses simples égales à $\Delta$ en tête de liste (s'il y a un $\Delta$, il
   est forcément en tête de liste). Incrémente le champ delta\_power en conséquence.

\ocweol
\label{Canonical.ml:4720}%
\medskip
\ocwbegincode{}\ocwindent{0.00em}
\ocwkw{let}~$\ocwlowerid{canonicize}~\ocwlowerid{bpl}~=$\ocweol
\ocwindent{1.00em}
\ocwkw{let}~$\ocwlowerid{dp}~=~$\ocwbt{ref}~$\ocwlowerid{bpl.}\ocwlowerid{delta\_power}~$\ocweol
\ocwindent{1.00em}
\ocwkw{and}~$\ocwlowerid{pl}~=~$\ocwbt{ref}~$(\ocwlowerid{make\_left\_weighted}~\ocwlowerid{bpl.}\ocwlowerid{permlist})$\ocweol
\ocwindent{1.00em}
\ocwkw{and}~$\ocwlowerid{ok}~=~$\ocwbt{ref}~\ocwkw{false}~\ocwkw{in}\ocweol
\ocwindent{1.00em}
\ocwkw{while}~$\lnot{}~!\ocwlowerid{ok}~\land{}~!\ocwlowerid{pl}~\not=~[\,]~$\ocwkw{do}\ocweol
\ocwindent{2.00em}
\ocwkw{if}~$\ocwupperid{P.}\ocwlowerid{is\_delta}~(\ocwupperid{List.}\ocwlowerid{hd}~!\ocwlowerid{pl})~$\ocwkw{then}~$($\ocweol
\ocwindent{3.00em}
$\ocwlowerid{dp}~:=~!\ocwlowerid{dp}~+~1;$\ocweol
\ocwindent{3.00em}
$\ocwlowerid{pl}~:=~\ocwupperid{List.}\ocwlowerid{tl}~!\ocwlowerid{pl}$\ocweol
\ocwindent{2.00em}
$)~$\ocwkw{else}~$($\ocweol
\ocwindent{3.00em}
$\ocwlowerid{ok}~:=~$\ocwkw{true}\ocweol
\ocwindent{2.00em}
$)$\ocweol
\ocwindent{1.00em}
\ocwkw{done};\ocweol
\ocwindent{1.00em}
\{$\ocwlowerid{bpl\_size}~=~\ocwlowerid{bpl.}\ocwlowerid{bpl\_size};~\ocwlowerid{delta\_power}~=~!\ocwlowerid{dp};~\ocwlowerid{permlist}~=~!\ocwlowerid{pl}\}$\medskip

\ocwendcode{}\ocwindent{0.00em}
Met une tresse décrite par un mot de tresse sous forme canonique.

\ocweol
\label{Canonical.ml:5147}%
\medskip
\ocwbegincode{}\ocwindent{0.00em}
\ocwkw{let}~$\ocwlowerid{canonical\_form}~\ocwlowerid{b}~=~\ocwlowerid{canonicize}~(\ocwlowerid{get\_permlist\_decomposition}~\ocwlowerid{b})$\ocweol
\ocwendcode{}\allowbreak\ocwsection
\label{Canonical.ml:5218}%
Tests d'égalité.

   Revient à mettre sous forme canonique, et les comparer.

\ocweol
\label{Canonical.ml:5304}%
\medskip
\ocwbegincode{}\ocwindent{0.00em}
\ocwkw{let}~$\ocwlowerid{braids\_equal}~\ocwlowerid{b1}~\ocwlowerid{b2}~=~(\ocwlowerid{canonical\_form}~\ocwlowerid{b1})~=~(\ocwlowerid{canonical\_form}~\ocwlowerid{b2})$\medskip

\label{Canonical.ml:5373}%
\ocwindent{0.00em}
\ocwkw{let}~$\ocwlowerid{permlists\_equal}~\ocwlowerid{bpl1}~\ocwlowerid{bpl2}~=~(\ocwlowerid{canonicize}~\ocwlowerid{bpl1})~=~(\ocwlowerid{canonicize}~\ocwlowerid{bpl2})$\ocweol
\ocwendcode{}\allowbreak\ocwsection
\label{Canonical.ml:5448}%
Opérations algébriques sur les tresses sous forme de listes de permutations.

\ocweol
\ocwindent{0.00em}
Produit.

   Formule : $(\Delta^p A_1...A_l)(\Delta^q B_1...B_l') =
               \Delta^{p+q} \tau^q(A_1)...\tau^q(A_l) B_1...B_l$

\ocweol
\label{Canonical.ml:5673}%
\medskip
\ocwbegincode{}\ocwindent{0.00em}
\ocwkw{let}~$\ocwlowerid{product}~\ocwlowerid{a}~\ocwlowerid{b}~=$\ocweol
\ocwindent{1.00em}
\{$\ocwlowerid{bpl\_size}~=~\ocwlowerid{a.}\ocwlowerid{bpl\_size};$\ocweol
\ocwindent{1.50em}
$\ocwlowerid{delta\_power}~=~\ocwlowerid{a.}\ocwlowerid{delta\_power}~+~\ocwlowerid{b.}\ocwlowerid{delta\_power};$\ocweol
\ocwindent{1.50em}
$\ocwlowerid{permlist}~=~($\ocwkw{if}~$\ocwlowerid{b.}\ocwlowerid{delta\_power}~$\ocwkw{mod}~2~$\not=~0~$\ocwkw{then}~$\ocwupperid{List.}\ocwlowerid{map}~\ocwupperid{P.}\ocwlowerid{tau}~\ocwlowerid{a.}\ocwlowerid{permlist}~$\ocwkw{else}~$\ocwlowerid{a.}\ocwlowerid{permlist})$\ocweol
\ocwindent{7.00em}
@~$\ocwlowerid{b.}\ocwlowerid{permlist}~\}$\medskip

\label{Canonical.ml:5888}%
\ocwindent{0.00em}
\ocwkw{let}~$(\ensuremath{<}*\ensuremath{>})~=~\ocwlowerid{product}$\medskip

\ocwendcode{}\ocwindent{0.00em}
Inverse.

   Formule : $(\Delta^r A_1...A_l)^{-1} =
               \Delta^{-r-l} \tau^{-r-l}(A'_l)...\tau^{-r-1}(A'_1)$
   avec $A'_i = A_i^{-1} \Delta$ en voyant $\Delta$ et $A_i$ en tant que permutations (??)

\ocweol
\label{Canonical.ml:6133}%
\medskip
\ocwbegincode{}\ocwindent{0.00em}
\ocwkw{let}~$\ocwlowerid{inverse}~\ocwlowerid{b}~=$\ocweol
\ocwindent{1.00em}
\ocwkw{let}~$\ocwlowerid{l}~=~\ocwupperid{List.}\ocwlowerid{length}~\ocwlowerid{b.}\ocwlowerid{permlist}~$\ocwkw{and}~$\ocwlowerid{q}~=~\ocwlowerid{b.}\ocwlowerid{delta\_power}~$\ocwkw{in}\ocweol
\ocwindent{1.00em}
\ocwkw{let}~$\ocwlowerid{delta}~=~\ocwupperid{P.}\ocwlowerid{make\_delta}~\ocwlowerid{b.}\ocwlowerid{bpl\_size}~$\ocwkw{in}\ocweol
\ocwindent{1.00em}
\ocwkw{let}~$(\ocwlowerid{\_},~\ocwlowerid{pl'})~=$\ocweol
\ocwindent{2.00em}
$\ocwupperid{List.}\ocwlowerid{fold\_left}~($\ocwkw{fun}~$(\ocwlowerid{parity},~\ocwlowerid{acc})~\ocwlowerid{p}~\rightarrow{}$\ocweol
\ocwindent{11.00em}
\ocwkw{let}~$\ocwlowerid{p'}~=~\ocwupperid{P.}\ocwlowerid{compose}~(\ocwupperid{P.}\ocwlowerid{inv}~\ocwlowerid{p})~\ocwlowerid{delta}~$\ocwkw{in}\ocweol
\ocwindent{11.00em}
$((\ocwlowerid{parity}~+~1)~$\ocwkw{mod}~2,\ocweol
\ocwindent{11.50em}
$($\ocwkw{if}~$\ocwlowerid{parity}~=~0~$\ocwkw{then}~$\ocwlowerid{p'}::\ocwlowerid{acc}~$\ocwkw{else}~$(\ocwupperid{P.}\ocwlowerid{tau}~\ocwlowerid{p'})::\ocwlowerid{acc})))$\ocweol
\ocwindent{9.50em}
$((\ocwlowerid{abs}~\ocwlowerid{q})~$\ocwkw{mod}~2,~$[\,])~\ocwlowerid{b.}\ocwlowerid{permlist}~$\ocwkw{in}\ocweol
\ocwindent{1.00em}
\{~$\ocwlowerid{bpl\_size}~=~\ocwlowerid{b.}\ocwlowerid{bpl\_size};$\ocweol
\ocwindent{2.00em}
$\ocwlowerid{delta\_power}~=~-~\ocwlowerid{q}~-~\ocwlowerid{l};$\ocweol
\ocwindent{2.00em}
$\ocwlowerid{permlist}~=~\ocwlowerid{pl'}$\ocweol
\ocwindent{1.00em}
\}\medskip

\ocwendcode{}\ocwindent{0.00em}
Conjugué.

   Conjugué de A par B : $B^{-1} A B$

\ocweol
\label{Canonical.ml:6677}%
\medskip
\ocwbegincode{}\ocwindent{0.00em}
\ocwkw{let}~$\ocwlowerid{conjugate}~\ocwlowerid{a}~\ocwlowerid{b}~=~(\ocwlowerid{inverse}~\ocwlowerid{b})~\ensuremath{<}*\ensuremath{>}~\ocwlowerid{a}~\ensuremath{<}*\ensuremath{>}~\ocwlowerid{b}$\ocweol
\ocwendcode{}\allowbreak\ocwsection
\label{Canonical.ml:6726}%
Convertit une tresse sous forme canonique en sa permutation 
    correspondante.
    Il s'agit d'un antimorphisme (souvent noté $\pi$).

\ocweol
\label{Canonical.ml:6869}%
\medskip
\ocwbegincode{}\ocwindent{0.00em}
\ocwkw{let}~$\ocwlowerid{braid2perm}~\ocwlowerid{b}~=~$\ocweol
\ocwindent{2.00em}
\ocwkw{let}~$\ocwlowerid{n}~=~\ocwlowerid{b.}\ocwlowerid{bpl\_size}~$\ocwkw{in}\ocweol
\ocwindent{2.00em}
\ocwkw{let}~$\ocwlowerid{perm}~=~($\ocwkw{if}~$\ocwlowerid{b.}\ocwlowerid{delta\_power}~$\ocwkw{mod}~2~=~0~\ocwkw{then}\ocweol
\ocwindent{10.00em}
$\ocwupperid{P.}\ocwlowerid{make\_id}~\ocwlowerid{n}$\ocweol
\ocwindent{8.00em}
\ocwkw{else}\ocweol
\ocwindent{10.00em}
$\ocwupperid{P.}\ocwlowerid{make\_delta}~\ocwlowerid{n}$\ocweol
\ocwindent{7.50em}
$)~$\ocwkw{in}\ocweol
\ocwindent{2.00em}
\ocwkw{let}~$\ocwlowerid{temp}~=~\ocwupperid{Array.}\ocwlowerid{make}~\ocwlowerid{n}~0~$\ocwkw{in}\ocweol
\ocwindent{2.00em}
$\ocwupperid{List.}\ocwlowerid{iter}~($\ocwkw{fun}~$\ocwlowerid{p}~\rightarrow{}~\ocwupperid{P.}\ocwlowerid{compose}~$\~{}$\ocwlowerid{dest}:\ocwlowerid{temp}~\ocwlowerid{p}~\ocwlowerid{perm};~\ocwupperid{Array.}\ocwlowerid{blit}~\ocwlowerid{temp}~0~\ocwlowerid{perm}~0~\ocwlowerid{n})~\ocwlowerid{b.}\ocwlowerid{permlist};$\ocweol
\ocwindent{2.00em}
$\ocwlowerid{perm}$\ocweol
\ocwendcode{}\allowbreak\ocwsection
\label{Canonical.ml:7211}%
Génère une tresse sous forme de liste de permutations aléatoire.
   À appeler avec l = DOUZE, n de l'ordre de 80 (en gros, $10^{1\ ou \ 2}$)

\ocweol
\label{Canonical.ml:7362}%
\medskip
\ocwbegincode{}\ocwindent{0.00em}
\ocwkw{let}~$\ocwlowerid{random\_braid\_sequence}~\ocwlowerid{n}~\ocwlowerid{l}~=$\ocweol
\ocwindent{1.00em}
\ocwkw{let}~\ocwkw{rec}~$\ocwlowerid{loop}~\ocwlowerid{acc}~=~$\ocwkw{function}\ocweol
\ocwindent{2.00em}
$\mid{}~0~\rightarrow{}~\ocwlowerid{acc}$\ocweol
\ocwindent{2.00em}
$\mid{}~\ocwlowerid{i}~\rightarrow{}~\ocwlowerid{loop}~(\ocwupperid{P.}\ocwlowerid{random\_permutation}~\ocwlowerid{n}~::~\ocwlowerid{acc})~(\ocwlowerid{i}-1)$\ocweol
\ocwindent{1.00em}
\ocwkw{in}\ocweol
\ocwindent{1.00em}
\{$\ocwlowerid{bpl\_size}~=~\ocwlowerid{n};~\ocwlowerid{delta\_power}~=~\ocwlowerid{l}-(\ocwupperid{Random.}\ocwlowerid{int}~(3\times{}\ocwlowerid{l}));~$\ocwbc{} tout à fait arbitraire \ocwec{}\ocweol
\ocwindent{1.50em}
$\ocwlowerid{permlist}~=~\ocwlowerid{loop}~[\,]~\ocwlowerid{l}\}$\ocweol
\ocwendcode{}\allowbreak\ocwsection
\label{Canonical.ml:7611}%
Affiche une tresse sous forme de permlist : (delta\_power | Permutations)

\ocweol
\label{Canonical.ml:7691}%
\medskip
\ocwbegincode{}\ocwindent{0.00em}
\ocwkw{let}~$\ocwlowerid{print\_braid\_permlist}~\ocwlowerid{bpl}~=$\ocweol
\ocwindent{1.00em}
$\ocwupperid{Printf.}\ocwlowerid{printf}~$\ocwstring{"(\%d\ocwvspace{}|\ocwvspace{}"}~$\ocwlowerid{bpl.}\ocwlowerid{delta\_power};$\ocweol
\ocwindent{1.00em}
$\ocwupperid{List.}\ocwlowerid{iter}~($\ocwkw{fun}~$\ocwlowerid{p}~\rightarrow{}~\ocwupperid{P.}\ocwlowerid{print\_permutation}~\ocwlowerid{p};~\ocwlowerid{print\_string}~$\ocwstring{"\ocwvspace{}"}$)~\ocwlowerid{bpl.}\ocwlowerid{permlist};$\ocweol
\ocwindent{1.00em}
$\ocwlowerid{print\_string}~$\ocwstring{")"}\ocweol
\ocwendcode{}\end{document}
